\newpage
\vspace*{\stretch{1}}

\printglossary[type=\acronymtype]
\label{sec:acronymes}
\setcounter{secnumdepth}{0}
\section{Lexique}
\begin{itemize}
    \item \textbf{Gros œuvre}: pour une maison représente l'ensemble des travaux participant à la stabilité et à la solidité du logement tel que la réalisation des fondations d'une maison. Il s'agit de travaux permettant à la construction de résister à son propre poids et aux agressions extérieures.
    \item \textbf{découpe plasma}: Le découpage plasma est un procédé de découpage par fusion localisée, dans lequel un jet de gaz ou d’air comprimé vient faire fondre instantanément le métal porté à une température de fusion. Une température proche de 18000 °C est alors générée par l'arc électrique créé entre la torche (embout de la machine pour la découpe) et la plaque de métal.
    \item \textbf{fraisage}: Procédé dans lequel un outil enlève de la matière par un mouvement rotatif. Comme pour le perçage, il est possible d'utiliser un large éventail d'outils de différents diamètres et de différentes duretés.
\end{itemize}
\vspace*{\stretch{1}}