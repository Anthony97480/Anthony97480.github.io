\rhead{\nouppercase{\leftmark}}
\pagenumbering{arabic} % Commencer la numérotation des pages
\vspace*{\stretch{1}}
\setcounter{secnumdepth}{0}
\begin{center}
    \section{Introduction}
\end{center}
J'ai réalisé ce stage en fin de première année à l'INSA de Toulouse, dans le cadre de l'un de mes enseignements en tant qu'élève-ingénieur. L'objectif de celui-ci était de me permettre de découvrir le monde de l'entreprise à travers une expérience du travail ouvrier. Ce stage de quatre semaines en entreprise s'est déroulé lors de l'été 2023, du 10 Juillet au 8 Août dans l'atelier de l'entreprise de construction métallique \acrshort{SMOI}*.\par

J'ai alors été supervisé par X.PAYET, le responsable de l'atelier. Et j'ai réalisé, dans ce secteur de la construction métallique, la fabrication de charpentes à travers de la découpe, de la lecture de plan...\newline

Avant de réaliser ce stage, je savais que cette entreprise était une S.A.R.L, travaillant dans le secteur de la construction métallique et que celle-ci suivait actuellement un plan de redressement. J'ai alors choisi cette entreprise de part son secteur d'activité mettant directement en pratique toute la théorie que j'apprends, mais aussi parce que celle-ci est une moyenne entreprise me permettant ainsi de pouvoir observer l'intégralité du fonctionnement de l'entreprise.\par

J'espérais par ce stage pouvoir découvrir comment était créer les charpentes métalliques mais aussi toute la théorie autour du fonctionnement des machines et de la lecture de plans. Je voulais donc pouvoir découvrir et apprendre plus sur les acquis que j'avais déjà et de pouvoir me forger une opinion sur ce métier dans l'atelier dont j'avais une appréhension par comparaison avec les cours de technique de l'ingénieur que j'avais eu en première année.\newline

Cependant la recherche de ce stage ne s'est pas passée sans difficulté. En effet, au départ je n'avais envoyé aucun e-mail à cette entreprise et ne connaissais même pas son existence. Je n'ai à la suite de mes recherches et envoi de CV reçu que des réponse négative, me permettant ainsi de découvrir la difficulté qu'est la recherche de stage et/ou d'emplois dans le milieux professionnel. J'avais alors comme contrainte de trouver un stage ouvrier et comme objectif personnel de trouver un stage me permettant de pouvoir appliquer toute la théorie apprise lors de mes études.\newline

Dans un premier temps, je vais présenter l'entreprise dans laquelle j'ai réalisé mon stage, puis détailler les tâches réalisées, et enfin je terminerais par une analyse de l'entreprise à travers mon activité, rôle de stagiaire et ressenti dans celle-ci.

 
\vspace*{\stretch{1}}